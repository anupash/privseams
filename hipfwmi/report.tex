\documentclass[a4paper,titlepage]{article}
\usepackage{fancyhdr}
\usepackage{a4wide}
\usepackage{moreverb}
\usepackage{graphicx}
\usepackage{longtable}
\usepackage{supertabular}
\usepackage{varioref}
\usepackage{hyperref}       % keep this as the last one
\usepackage{acronym}
\begin{document}

\title{HIP Firewall Management Interface}

\author{
	Juha-Matti Tapio, \texttt{jmtapio@cs.Helsinki.FI} \\
	Helsinki University of Technology: \\
	Telecommunications Software and Multimedia Laboratory}
\maketitle

\fancyhf{} % Clear all fields
\fancyhead[C]{\small HIP Firewall Management Interface}
\fancyfoot[C]{ \thepage }
\pagestyle{fancy}

\tableofcontents

\newpage

\section*{Acknowledgements}

Todo.

\newpage

\begin{abstract}

Abstract.

\end{abstract}

\clearpage


\section{Introduction}



\section{Background}

\subsection{Host Identity Protocol}

\subsection{HIP Firewall}



\section{Implementation Architecture}

The HIP Firewall Management Interface is a proof of concept software
for controlling multiple HIP firewalls centrally. Its goal is to make
it easier for network administrators to change firewall policy without
the need to individually log in to each firewall host and to use
command line tools to interface with the firewall.

The Management Interface consists of two separate parts. Firewall
Controller and Management Logic are run together as daemons at each of
the firewall hosts. They implement the firewall configuration changes
that the administrator requests. Management Console on the other hand
is placed on the organization's web server and it presents the
administrator with the web user interface.

\subsection{Firewall Controller}

Firewall Controller is run at each of the firewall hosts. It is run
with root privileges and its main function is to reload the HIP
firewall when the configuration has changed. Firewall Controller
spawns Management Logic as a separate process for processing the
incoming configuration requests. Management Logic's functionality has
been split off from Firewall Controller so that request processing can
be done without root privileges and therefore with smaller security
risks should Management Logic be compromised for some reason.

Firewall Controller behaves like a normal UNIX daemon. It double forks
and detaches into the background and logs about events via syslog.

Access control to Management Logic is out of this software's
scope. The HIP firewall should be used to grant access only from the
host running Management Console.

\subsection{Management Logic}

Management Logic accepts configuration connections on a tcp socket.
Configuration requests are received via a simple XML message passing
protocol. Management Logic is responsible for maintaining the HIP
firewall's rulefile and source keys.

\subsection{Management Console}

Management Console has been implemented as a CGI-script that connects
to the individual firewall hosts' Management Logic -instances and
presents a HTML user interface to the administrator. Management
Console authenticates the administrator with HTTP Basic Authentication
\cite{rfc2617}.

It should be noted that Management Console targets HTML
4.01\cite{html401} and aims to be standard compliant. Therefore it
uses even parts of the HTML standard (especially the
$<$\texttt{button}$>$-tag) that Microsoft Internet Explorer does not
implement at the time of writing. If need arises to implement
workarounds, the best way to do that seems to create special HTML
templates where $<$\texttt{button}$>$-tags are converted into
$<$\texttt{input type="submit"}$>$-tags with separate forms. Such
workarounds have not been implemented so far because it is difficult
to create them without breaking some of the limits the HTML standard
places on the location of $<$\texttt{form}$>$-tags.

\subsection{Configuration Protocol}

Management Console and Management Logic communicate with a custom
message passing protocol. Management Console acts as a server and
Management Logic acts as the client. Each connection consists of two
phases: 1) the client sends a complete well formed XML request message
to the server and 2) the server replies with another complete well
formed XML response message.

\subsubsection*{Possible Request Elements}

\begin{description}
\item \texttt{query} Main element containing all the parts of the request. \\
	Must have attribute \texttt{protoversion=0.1}.
\item \texttt{list\_rules} Ask for a list of the current rules.
\item \texttt{empty\_rules} Delete all existing rules.
\item \texttt{add\_rules} Append the included rules to the current rules.
\item \texttt{remove\_rules} Remove rules matching the included rules.
\item \texttt{list\_keys} Ask for a list of keys stored.
\item \texttt{upload\_key} Upload a key contained within this element to the keystore. \\
	Must have attribute \texttt{name} for filename.
\item \texttt{delete\_key} Delete a key from the keystore. \\
	Must have attribute \texttt{name} for filename.
\item \texttt{echo} Echo contents of this request. For testing.
\end{description}

\subsubsection*{Possible Reply Elements}

\begin{description}
\item \texttt{results} Main element containing all the replies. \\
	Must have attribute \texttt{protoversion=0.1}.
\item \texttt{emptied\_rules} Existing rules were removed.
\item \texttt{removed\_rules} Some rules were removed. \\
	The count of removed rules is reported with attribute \texttt{count}.
\item \texttt{added\_rules} Some rules were added. \\
	The count of added rules is reported with attribute \texttt{count}.
\item \texttt{list\_rules} Includes the current rules.
\item \texttt{list\_keys} Includes the names of keys stored.
\item \texttt{key} In \texttt{list\_keys} mentions a single key. \\
	The name of the key is reported with attribute \texttt{name}.
\item \texttt{echo} Repeats the contents of the \texttt{echo}-request. For testing.
\end{description}

\subsubsection*{Rule Elements}

Rule elements specify the rules in XML. They can be used within
\texttt{add\_rules} and \texttt{remove\_rules} for requests and within
\texttt{list\_rules} for replies.

All subelements have the \texttt{not} attribute. It must contain
either ``0'' or ``1'' where the latter reverses the meaning of the
element's main attribute. I.e. ``1'' corresponds to the ``!'' modifier
of HIP firewall's native rule syntax.

\begin{description}
\item \texttt{rule} A single rule specification. \\
	Must have attributes \texttt{hook} and \texttt{target}. \\
	May contain at most one of each of the following elements.
\item \texttt{src\_hit} Source HIT. \\
	Must have attributes \texttt{not} and \texttt{hit}.
\item \texttt{src\_hi} Source HI. \\
	Must have attributes \texttt{not} and \texttt{hi}. \\
	\texttt{hi} should match the name of a stored key.
\item \texttt{dst\_hit} Destination HIT. \\
	Must have attributes \texttt{not} and \texttt{hit}.
\item \texttt{pkt\_type} Packet type. \\
	Must have attributes \texttt{not} and \texttt{type}.
\item \texttt{in\_iface} Incoming interface. \\
	Must have attributes \texttt{not} and \texttt{iface}.
\item \texttt{out\_iface} Outgoing interface. \\
	Must have attributes \texttt{not} and \texttt{iface}.
\item \texttt{state} Connection state. \\
	Must have attributes \texttt{not}, \texttt{state}, \texttt{vrfy\_resp} and \texttt{acpt\_mobile}.
\end{description}

\subsubsection*{Example request}

\begin{verbatim}
<?xml version="1.0"?>
<query protoversion="0.1">
  <empty_rules/>
  <add_rules>
    <rule hook="FORWARD" target="ACCEPT">
      <in_iface not="0" iface="eth0"/>
    </rule>
    <rule hook="OUTPUT" target="ACCEPT"/>
    <rule hook="INPUT" target="ACCEPT">
      <state not="0" state="ESTABLISHED" vrfy_resp="0" acpt_mobile="1"/>
    </rule>
  </add_rules>
  <list_rules/>
</query>
\end{verbatim}

\subsubsection*{Example reply}

\begin{verbatim}
<?xml version="1.0"?>
<results protoversion="0.1">
  <emptied_rules/>
  <added_rules count="3"/>
  <list_rules>
    <rule hook="FORWARD" target="ACCEPT">
      <in_iface not="0" iface="eth0"/>
    </rule>
    <rule hook="OUTPUT" target="ACCEPT"/>
    <rule hook="INPUT" target="ACCEPT">
      <state not="0" state="ESTABLISHED" vrfy_resp="0" acpt_mobile="1"/>
    </rule>
  </list_rules>
</results>
\end{verbatim}

\section{Usage}

Parts of the management interface need to run on all of the managed
firewall hosts and on some web server. All hosts need to support HIP
because Management Console (on the web server) uses HIP to communicate
with Management Logic (on the firewall host).

Configuration information is mostly stored in the configuration file
\texttt{hipmi.conf} which should exist in management interface's
current working directory. The file has been divided into sections
denoted by brackets. The sections that are currently in use are
\texttt{daemon} and \texttt{hipfirewall}. The first is used for configuring 
Firewall Controller's behaviour and the latter is used to describe how
the HIP Firewall has been configured to work.

\subsubsection*{Example \texttt{hipmi.conf}}

\begin{verbatim}
[daemon]
interface: ::1
port: 11235
pidfile: /var/run/hipmid.pid
pid: 1000
gid: 1000

[hipfirewall]
reloadcommand: /etc/init.d/hipfw force-reload
rulefile: /var/lib/hipfw/hipfw-rules
keydirectory: /var/lib/hipfw/keys
\end{verbatim}

\subsection{On the firewall host}

todo

\subsection{On the web server}

todo

\section{Conclusion}


\bibliographystyle{plain}
\bibliography{report}

\appendix
\newpage

\end{document}
