\appchapter{Design Choices Catalogue}
\label{sec:catalogue}

% Design choice catalogue (rules of thumb):
%
% - Do not concentrate on architectural details.
% - Put more stuff about decisions here: why have we done things
%   using this way and not that way, even subconsciously. The decision
%   may have been noticed immediately or afterwards? Include all decisions
%   and especially those which made us change our minds afterwards. -pnr

\appsection{Kernel Model}
\label{sec:kernel_model_design_choices}

\begin{designchoice}{Distribution of intelligence between the kernel module
and the userspace daemon}

\alt{Intelligent daemon or intelligent kernel module}

\select{Intelligent kernel module}

\record{20020312}

\background{The overall functionality needs to be distributed between
the daemon and the kernel module. Intelligent kernel module approach
means that most of the responsibility is in the kernel and the daemon
is used as a ``slave'' of the kernel. Intelligent daemon means that
the daemon maintains the state machine and the kernel is used merely
as a packet forwarder.}

\comparison{A short comparison of the two models of operation is
listed in \autoref{tbl:architecture_comparison}.
\\&
The kernel based implementation tries to handle incoming and outgoing
packets in the kernel and calls the daemon only when needed. The
packets in the base exchange require many cryptographic operations
that cannot be handled in the kernel. This implies many ``small''
exchanges of information between the kernel and the daemon to process
a single HIP header. Many context switches occur because of many small
messages between the userspace and the kernel. The time required for a
context switch in the Linux kernel is known to be lower than in many other
operating systems (\cite{unixbenchmark}).
\\&
The small exchanges of information in the kernel based implementation
imply ``bump-in-the-stack'' code in the kernel, meaning that the control
flow bounces back and forth between userspace and kernelspace. The
chained code requires saving and restoring of the current state of the
HIP header being processed. The chained code is a bit awkward to use but
it is necessary in the kernel based approach.
\\&
On the other hand, the daemon based implementation captures HIP
headers in the kernel and forwards them as they are to the daemon for
processing. This implies just a few ``big'' messages between the
userspace and the daemon to process a HIP header.
\\&
In the kernel based implementation, the daemon has no state in the
kernel because the state is stored in the kernel. The daemon based
implementation stores the state in the daemon. The choice of the model
of operation strongly affects the interface between the daemon and
the kernel.
\\&
\textit{PF\_KEY} is a standard protocol to exchange cryptographic
key information between the userspace and the kernel. The kernel based
implementation does not need to communicate \textit{PF\_KEY} related data to
the daemon because it can be done directly inside the kernel.
\\&
One practical property is not listed in
\autoref{tbl:architecture_comparison} but is still worth mentioning. It is
related to the actual development process itself: debugging is usually
considered to be easier in the userspace than in the kernel. The
source of a bug is harder to trace if the operating system crashes
before debugging messages appear on a terminal or a disk device.}

\revision{On 20031016, the kernel based approach was revised. The DSA and
Diffie-Hellman functionality, that was used in the HIP daemon, were
ported into kernelspace. This obviates the need for userspace
callbacks. The most important benifit is that the kernel module code
could be simplified because the control flow is more linear in this
style.}

\end{designchoice}

\bottomcaption{Comparison of kernel based and userspace based HIP
               architectures}
\label{tbl:architecture_comparison}
\begin{supertabular}{|p{0.6\textwidth}|p{0.15\textwidth}|p{0.15\textwidth}|} \hline
\textbf{Property} & \textbf{Kernel based} & \textbf{Userspace based}\\\hline
communication between userspace and kernel is minimized & no & yes \\\hline
amount of context switches is minimized & no & yes \\\hline
stateless daemon & yes & no \\\hline
\textit{PF\_KEY} messages are sent between kernel and userspace & no & yes \\\hline
\end{supertabular}

\appsection{Networking Stack}
\label{sec:networking_stack_design_choices}

\begin{designchoice}{Interface to networking stack}

\alt{Refactoring of networking stack or hooks}

\select{Hooks}

\record{20020316}

\background{
Originally there were two options for interfacing with the Linux
networking stack. One option was to refactor the Linux networking stack to
have a unified interface between the network and the transport layers and
then to install a new layer between. The other option was to place
``hooks'' in networking stack that would capture both inbound and
outbound network traffic. The layout of hooking model is visualized in
\autoref{fig:input_hooks} and \autoref{fig:output_hooks}.
}

\comparison{ The interface with the Linux networking stack was
selected to be based on the hooks for two reasons. First reason was
that unifying the interfaces by refactoring did not succeed because
it would have been too time consuming. The second reason was that
MIPL \cite{mipl} project had already had positive experience with
hooks in their experimentations with the Linux networking stack.  }

\revision{The implementation was revised by moving the hooks from the
TCP layer to the IPv6 layer. It simplified the implementation and UDP
support was gained with no extra effort.}

\revision{20040501 Some of the hooks were changed due to transition to the
          Linux kernel 2.6 series.}

\end{designchoice}

% XX FIXME: Add Kristian's thesis stuff here
% - implementation.tex, \section{HIP modifications}, summary

\appsection{Sockets API}

\begin{designchoice}{HIT socket structure}

\alt{\textit{sockaddr\_hip} or \textit{sockaddr\_in6}}

\select{\textit{sockaddr\_in6}}

\record{20020730}

\background{ Two different alternatives to store HITs were
considered. First alternative was to recycle the structure used for
storing IPv6 addresses, \textit{sockaddr\_in6}, because HIT address
size equals IPv6 address size. The second alternative was to create a
new structure, \textit{sockaddr\_hip}, for storing HITs. The new
structure had also some extra space to store a mapping to an IPv6
address. }

\comparison{

\textit{sockaddr\_hip} was implemented in the beginning of the
project. A mapping to an IPv6 address was stored in the same
structure. The benefit of this alternative was that the mapping was
communicated to the kernel directly in a \textit{connect} call and no
separate mechanism was needed for communicating the mapping to the
kernel.
\\&
\textit{sockaddr\_hip} was later dropped in favor of recycling
\textit{sockaddr\_in6}. \textit{sockaddr\_hip} was considered a bad
way to store HITs. Transparent mode (see
page~\pageref{sec:api_architecture}) would have been virtually impossible to
implement if \textit{sockaddr\_hip} had been used. }

\revision{20040901 Native HIP API \cite{hipapithesis} was finished. It is the
          cleanest way to use the sockets API.}

\end{designchoice}

\appsection{Mappings}

\begin{designchoice}{Poisoning of mapping tables}

\alt{Mappings are communicated to the kernel directly or through the daemon}

\select{Mappings are communicated directly to kernel}

\record{20021212}

\background{ There is a local security problem related to
communicating HIT-to-IPv6 mappings to the kernel using the resolver (see
page~\pageref{sec:mapping_architecture}). All users must be allowed to
communicate mappings with the resolver to the kernel and this could be
exploited by a local attacker in the system. The local attacker could
forge some mapping messages and poison the mapping tables in the kernel.
\\&
The resolver could send mapping messages to the daemon and daemon
could verify the validity of mappings from DNS. This way, kernel would
receive only valid mappings. The other alternative is to send mappings
directly to kernel and tag them with the UID of the sender of the mapping
to prevent poisoning of the kernel mapping tables.  }

\comparison{

The idea of communicating mappings through daemon was dropped
basically because it was inefficient and more complicated to
implement. Inefficiency is consequence of extra context switches and
an extra call to the resolver. The extra context switches are required
because the daemon is an intermediary between the kernel and the
resolver. An extra call to the resolver is needed to validate the mapping
from DNS. A new interface would have had to be opened in the daemon, for
example an UNIX domain socket, to listen for incoming mappings.
\\&
Communicating the mappings directly to the kernel was considered more
efficient and easier to implement. The overhead of the tagging
alternative was considered to be small. Direct communication with the
kernel can then use existing interfaces that the daemon already
used. Creating a new interface for listening for incoming mappings can
be also avoided in the daemon. }

%\revision{xx}

\end{designchoice}

% XX TODO: Miika, fix this. Add also the HIP socket option.
%
\begin{designchoice}{Userspace messaging format}

\alt{Extended HIP TLV format, extended \textit{PF\_KEY} format or in-house}

\select{Extended HIP TLV format}

\record{20020723}

\background{ The daemon and the kernel need to communicate information
somehow. Four different alternatives were considered. The first
alternative, \textit{PF\_KEY}, was already present in USAGI and it was
mainly used for exchanging cryptographic key information with the
kernel and an userspace IKE daemon. The second alternative, HIP packet
format \cite{hip}, could also be recycled for this task. Both
\textit{PF\_KEY} and HIP packet format would have required some
extensioning to fit for the job. The third option would have been to
invent an ``in-house'' format optimized for the task. The fourth
alternative would have been \textit{NETLINK} \cite{netlink}.}

\comparison{ An in-house format specialized to sharing information
between the kernel and the userspace could have been more efficient
and compact than the other alternatives. It was mainly dropped because
it seemed that it was not worth to reinvent the wheel. The other
alternatives were more attractive because they could be used with just
some minor extensions to accommodate the task.
\\&
\textit{PF\_KEY} was considered for two reasons in the beginning of
the project: it was already available in the USAGI project and there
seemed to be a need to communicate key material related to IPSec
between the HIP daemon and the kernel. \textit{PF\_KEY} was dropped
because it would have needed too many changes to accommodate for the
task because it was just used for exchanging key information. It was
also noticed later when base exchange was almost finished that IPSec
related \textit{PF\_KEY} keying operations could be done directly
inside the kernel. The reason for this was the kernel-oriented model
that was discussed in \autoref{sec:kernel_model_architecture}.
\\&
\textit{NETLINK} can only be used for transferring network packets
between the userspace and the kernel. It was rejected in the design
for two reasons. First, the userspace daemon does not need to know
anything about the network packets headers, such as IPv6 headers, and
the unneeded headers would also cause more overhead. Second, the
request-response scheme used in the daemon architecture requires also
something more than just the packet headers.
\\&
HIP packet format was selected for two reasons. First, the information
communicated between the kernel and the userspace contained mainly the
same parameters used in HIP \cite{hip}. The second reason is a
consequence of the first one: it would have been silly to develop
separate message builders and parsers for kernel-to-daemon and
peer-to-peer communication if the message format was essentially the
same. One single message builder/parser saves development time and
increases code quality.
}

\revision{20041101 The userspace applications communication using
          \texttt{PF\_HIP} sockets with the kernelspace.}

\end{designchoice}
